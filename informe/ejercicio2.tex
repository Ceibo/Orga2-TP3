\subsection{Ejercicio 2}
\subsubsection{Rutinas de atención a las excepciones del procesador}
Cada interrupción debe tener su entrada en la Tabla de Descriptores de Interrupción (IDT). Las primeras 32 posiciones de esta tabla (índices 0 a 31) corresponden a las excepciones del procesador.\newline
Así como en la GDT, para las entradas de la IDT utilizamos un struct de C, cuyos campos y valores son los siguientes:
\begin{itemize}
\item Selector de segmento: 0x40 que corresponde al índice del segmento de código del kernel en la GDT corrido 3 bits a la izquierda (índice 8).
\item Desplazamiento: La dirección donde comienza el código de la rutina de atención. Es una dirección distinta para cada rutina. Las rutinas están definidas con una macro en ensamblador cuyas etiquetas están declaradas como símbolos globales para poder referenciarlas luego desde C.
\item Atributo: Es un campo que agrupa varios valores. En nuestro caso el valor es 0x8E00. El 8 es un valor hexadecimal que en binario es 1000. El 1 es el bit de presente, los dos bits siguientes son el DPL (nivel 00) y el bit menos significativo debe estar en cero para interrupciones. La E también es un valor hexadecimal que corresponde al $gate type$ y en este caso significa puerta de interrupción de 32 bits.  Así, el valor de atributo es 0x8E y luego se agregan dos ceros a la derecha porque debe estar corrido siempre 8 bits (los ceros son fijos).
\end{itemize}
Las entradas de la IDT se cargan desde una dirección de memoria arbitraria. Una vez cargadas, inicializamos la IDT utilizando la instrucción lidt (load IDT).\newline
Las rutinas solo muestran el número de excepción (que corresponde al índice de la IDT) y luego quedan en un ciclo infinito.
