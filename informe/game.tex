\subsection{Ejercicio 7: Estructuras y Lógica De Juego}
\subsubsection{Estructuras}


Para armar la lógica del juego tenemos entre varias estructuras la de $pirata$, $jugador$, $sched$ $task$ y $sched$.
\begin{itemize}
\item La estructura de $jugador$ tiene los siguientes campos: $indice$ que indica si es jugador $A$ o $B$, 
array $piratas$ de tamaño ocho que contiene punteros a los piratas del jugador, dos arrays $vistas$ $x$ y
$vistas$ $y$ que contienen posiciones vistas por pirata de jugador en cada turno. Luego sigue campo
$puntos$, para guardar los puntos del jugador y campos $puerto$ $x$ y $puerto$ $y$, con posiciones iniciales
en el mapa de los piratas del jugador.\\

\item La estructura de $pirata$ tiene los siguientes campos: $id$ que indica en que índice de la $gdt$ está el
descriptor de su $tss$ (no es selector); $index$, con valores entre 0 y 7, que sirve para indizar en el array de 
piratas de jugador dueño; $jugador$ que es puntero a pirata dueño; $libre$ que es flag que indica si
el slot en array $piratas$ de jugador asociado al pirata está libre; $x$ e $y$ que indican la posición del
pirata en el mapa; $tipo$ que indica el tipo de pirata (minero o explorador) y $reloj$ indicando reloj del pirata.
\\
\item La estructura $sched$ tiene los siguientes campos: $tasks$ que es array de estructuras $sched$ $task$ $t$ y tiene
tamaño 17, 16 tareas de jugadores más tarea idle; $current$ indicando índice de tarea actual en $tasks$; 
campos $posiciones$ $tesoros$ $A$ y $posiciones$ $tesoros$ $B$ que son colas $fifo$ (first in first out) de 
tuplas $x$ e $y$ y almacenan posiciones con monedas descubiertas por cada jugador; $inicio$ $tesoros$ $A$ y
$inicio$ $tesoros$ $B$ que indican primer lugar ocupado de las colas de posiciones anteriores (al consultar
posiciones de tesoros se chequean estos índices y se sacan usando su valor; luego se incrementan en uno apuntando
al siguiente elemento de la cola y en caso de llegar al final de la cola apuntará al principio); campo $prox$
que indica índice en $tasks$ de siguiente tarea a la actual y por último $tiempo$ $sin$ $cambios$ indicando 
el tiempo de juego en que ningún jugador anotó puntos (sirve para finalizar juego).
\\
\item La estructura de $sched$ $task$ tiene los siguientes campos: $gdt$ $index$ que indica selector de $tss$ asociado
a tarea en ese slot de array $tasks$, $pirata$ con puntero a tarea del $sched$ $task$, flags
$reservado$ $minero$ y $reservado$ $explorador$ tal que al llegar scheduler a este $sched$ $task$ indican si 
se debe activar la tarea y que tipo darle, $posiciOn$ $x$ $tesoro$ y $posiciOn$ $y$ $tesoro$ que en caso de 
que se deba activar minero indican a donde buscar las monedas.
\end{itemize}

\subsubsection{Lógica del juego}
 
Al iniciar juego se inicializan los jugadores asignándoles índice ($A$ o $B$), se setean puntos en 0, 
inicializamos el mapa de cada jugador, pedimos las páginas libres para mapear las posiciones descubiertas y ser 
accedidas por todos los piratas de cada jugador (las usan los mineros para moverse). Luego inicializamos los 
piratas.\\
Al iniciar piratas se asignan a los piratas de cada jugador (8 para cada jugador) un número $id$ que no cambia en 
el resto del juego. El $id$ asocia a cada pirata con el índice del descriptor de su $tss$ en la $gdt$ (los $id$ 
de los piratas de jugador $A$ están en rango desde 15 a 22 y los de jugador $B$ en rango desde 23 a 31). Entonces
el scheduler puede ubicar el descriptor de $tss$ del pirata a través de su $id$ y usarlo para cambiar de tarea
(shifteando en tres a izquierda al $id$ y sumando atributos obtenemos selector en $gdt$). Es decir que 
inicialmente cada pirata de jugador tiene descriptor de $tss$ inicializada (asociada a una
$tss$ vacía). Luego a los piratas se les asigna además del $id$ un índice, entre 0 y 7, para iterar en array
de piratas de cada jugador. También se les asigna un puntero al jugador dueño y se les setea el campo $libre$ 
en 1 ($true$) y campo $reloj$ en 0.\\
    
Luego se inicializa el scheduler, se asigna a cada posición de array $tasks$ (17 en total) un $sched$ $task$ 
asociado a tarea, fijando en posiciones dadas por índice 1 a 8 tareas de $A$ y posiciones en índice 9 a 16 
tareas de $B$; en índice 0 va la tarea $idle$.
Se setean atributos de cada $sched$ $task$ seteando índice en gdt de su tarea, puntero a pirata de jugador, los 
flags $reservado$ $explorador$ y $reservado$ $minero$ se setean en falso y parámetros de posiciones de tesoros se 
ponen en 100 (vacías). 
Como en posición 0 de array $tasks$ va la $idle$ al inicio el parámetro $current$ de scheduler
 se setea en 0 (es la única activa).
Luego se deben inicializar los arrays $posiciones$ $tesoros$ con tuplas de $x$ e $y$ en 100 indicando 
posiciones vacías (con un máximo de 20 posiciones disponibles). También seteamos $prox$ en 500, indicando que
la siguiente no es tarea de jugador.\\

Luego de inicializar todo la única tarea activa es la $idle$. Si queremos lanzar piratas debemos usar las 
interrupciones.

\subsubsection{Interrupción de teclado}
Al pulsar tecla de lanzar pirata ($shift L$ o $shift R$) se chequea en piratas de jugador si hay slot libre
tal que en ese caso se debe setear en su $sched$ $task$ el campo $reservado$ $explorador$ como $true$ y 
llenamos 
los campos del pirata: $libre$ en falso, seteamos su posición con valores de puerto de jugador y le asignamos
tipo explorador. No inicializamos su tss porque si activaramos tarea en interrupción de reloj se puede pisar 
código en switch de tarea (puede suceder que siguiente tarea salga del mismo puerto y no sea explorador.
Luego en esa posición física del mapa se pisará el código del explorador). Sin embargo si hay algún pirata
en el puerto y hay slot libre se lanzará nuevo pirata pisandose el código del primero.\\
Se debe chequear campo $prox$ de scheduler tal que si la tarea dada por $prox$ pertenece a mismo jugador que
la tarea actual, la de $current$, y pirata lanzado es de jugador contrario a estos se debe 
pisar $prox$ con índice de nueva tarea. Caso contrario (nueva tarea pertenece a jugador actual) no se toca $prox$.\\

En caso de pulsar la tecla $y$ se activará el modo debug. Para esto usamos los flags $modo$ $debug$ y
$pantalla$ $debug$ $activada$ que nos indican si el modo está activo o no. En caso de no activo al presionar
tecla $y$ se setea $modo$ $debug$ en $true$ y ante cualquier excepción se mostrará una pantalla con información
de los registros de la tarea que murió. Estos son: registros generales, que se obtienen directamente en handler de 
excepción; el $eip$, se obtiene buceando en la pila, los segmentos de código, los eflags y registros de control. Todos se almacenan en estructura auxiliar
llamada $debugger$ tal que al tener que imprimirse por pantalla estos valores los sacamos de ahí.\\
Antes de mostrar la información por pantalla se debe resguardar pantalla actual de juego. Para esto usamos
una matriz auxiliar de tamaño de la pantalla tal que al salir de modo debug de allí obtenemos los datos para 
pintar la pantalla que se mostraba justo antes de la excepción. Este modo estará activo hasta que se presione
la $y$ nuevamente en el juego. Durante tiempo en que está activo no se permite mover ni lanzar pirata.\\

\subsubsection{Interrupción de reloj}
En cada interrupción de reloj se consulta si el debugger está activo y si es así no se debe cambiar de tarea.
Si debug no está activo llamamos $sched$ $atender$ $tick$. Esta función hace lo siguiente: actualiza el
reloj global, el reloj del pirata actual y en caso de que $scheduler$ $tiempo$ $sin$ $cambios$ sea igual a
10000 se dará por terminado el juego.\\
Luego se chequea si hay slot de algún jugador seteado como libre
(se chequean para cada jugador). 
Si hay slot libre buscamos tarea en espera, es decir, se debe chequear en array $posiciones$ $tesoros$ de 
scheduler asociado a jugador dueño de slot libre tal que si encuentra tupla con valores distinto de 100 (moneda descubierta) entonces 
en estructura $sched$ $task$ del pirata libre se debe setear campo $minero$ en $true$ y copiar valores de tupla, $x$ e $y$, 
a $posicion$ $x$ $tesoro$ y $posicion$ $y$ $tesoro$ de $sched$ $task$. Luego se restauran esos valores de
 $posiciones$ $tesoros$ con 100 (indicando posición vacía) para próximas posiciones a guardar.
Esto se debe repetir por cada slot vacío que se encuentre, es decir chequear si hay minero en espera por cada 
slot vacío para cada jugador.\\

Si campo $prox$ de scheduler está apuntando a tarea activa entonces actualizamos $current$ con ese $prox$ para
switchear allí.\\
 
Luego buscamos siguiente tarea perteneciente al jugador reciente y actualizamos parámetro $prox$ de scheduler tal 
que en próximo switch de tareas se use esta tarea. Si no hay debe actualizarse $prox$ con siguiente de 
jugador contrario a jugador reciente. En caso de no haber jugadores activos $prox$ no se actualiza.\\

Si la tarea a switchear (dada por $current$ actualizado) estuviera reservada se debe activar pirata, inicializar 
su $tss$ (incluye mapear su posición en mapa) y pintar. Luego se debe setear campo de reservado en falso.\\
Si se reservo para minero se debe guardar en su pila de nivel $3$ los valores de campo $posicion$ $x$ $tesoro$, 
$posicion$ $y$ $tesoro$ (posiciones donde cavar) y lanzar al minero. Esto se hace mapeando una página temporal apuntando a la pila y luego de copiar los datos la desmapeamos.\\
Si estuviera reservado para explorador limpiamos el flag de reservado y lanzamos la tarea.
Devolvemos el selector de tarea siguiente (dado por $current$ recientemente actualizado).\\

Devuelta en handler de interrupción de reloj, con selector de siguiente tarea en $ax$, cargamos el selector actual
con instrucción $str$ en registro $cx$ y comparamos $ax$ con $cx$. Si son distintos hacemos un jump $far$ a
la siguiente tarea. Si son iguales se debe salir de la interrupción sin switchear. 


\subsubsection{Interrupción de software $0x46$}
Se chequea que caso de syscall se requiere. 
\begin{itemize}
\item Al mover:\\
Se chequea que la posición nueva sea válida (dentro del mapa). Si no es así se mata la tarea.
Si la nueva posición es válida se mueve su código a nueva posición en mapa. Para esto se mapea una página no usada por los jugadores a la anterior posición del pirata, luego se usa esta página para acceder al código de la tarea 
y se lo copia a la nueva posición, que ya está mapeada (hecho al inicializar $tss$). Finalmente se desmapea
esta página temporal.\\
Luego si tipo de pirata es explorador chequeamos si en las posiciones descubiertas válidas (3 celdas en dirección 
a mover) hay monedas. En caso de haber descubierto tesoro se busca tarea libre de jugador actual y en
estructura $sched$ $task$ asociada al pirata se reserva como minero, guardando la posición del tesoro
en campos $posicion$ $x$ $tesoro$, el $x$, $posicion$ $y$ $tesoro$, el $y$, para que el scheduler cuando llegue a 
ese slot lo active y lo mande a esos lugares.\\
Si no hay slot disponible para lanzar minero entonces guardar posición de tesoro descubierta 
para cuando alguno se libere. Esto se almacena en array $posiciones$ $tesoros$ de $sched$ $task$ correspondiente 
a jugador llamador. El scheduler ante primer interrupción se encargará de chequear si hay
lugar y entonces lanzar minero. Si no hay posiciones libres en $posiciones$ $tesoros$ las monedas descubiertas se descartan.\\
Si tipo de pirata es minero se chequea que su nueva posición esté mapeada. Esto se hace comprobando que la nueva 
posición esté presente como dirección virtual, es decir que esté mapeada como $read$ $only$ entre las páginas mapeadas por los exploradores de un jugador. Si no es así se mata al pirata.
	
\item Al cavar:\\
Se chequea que no se invoque con parámetros inválidos (campo $id$, tipo de tarea). Luego consultamos si hay 
monedas en la posición actual y cuántas hay. Si hay cero entonces se debe matar la tarea. Si hay una o más 
monedas se incrementa puntaje de jugador dueño de minero en uno y se decrementa cantidad de monedas en esa
posición. 

\item Consultar posición:\\
Chequeamos que los parámetros sean válidos ($id$ de pirata válido y número de caso en rango dado por -1 y 7). 
Luego retornamos la posición requerida codificada en un $uint32$. Como tenemos que devolver la posición al 
llamador y la devolvemos por registro $eax$, a la hora de desarmar el stack pointer, cuando debemos
restaurar el valor de $eax$ ignoramos el valor pusheado de eax incrementando el $esp$ en 4.
 

\end{itemize}


En todos los casos al final del procedimiento se debe saltar a la tarea $idle$.  


\subsubsection{Excepciones (0 a 19)}
Se mata a la tarea que produjo la excepción. Acá vienen a parar las tareas que realicen movimiento inválido,
llamen a syscall con parámetros incorrectos y tiren las demás excepciones del procesador.\\
Cuando la tarea muere se chequea si $prox$ de scheduler apunta a esta (tarea única). En ese 
caso se debe setear $prox$ en default (500, ninguna tarea activa) y no se debe modificar campo $current$ (por más 
que el pirata actual muera). También seteamos campo $libre$ del pirata con 1 y si el modo debug está activo 
mostramos la información del pirata por pantalla, esto es, registros de uso general, $eflags$,
registros de control y segmentos de código.\\
Antes de salir del handler habilitamos las interrupciones, con instrucción $sti$, y luego saltamos a la tarea
$idle$. En este caso no conservamos los registros de la tarea porque se la desaloja.


 

