Un sistema operativo es el conjunto de programas de un dispositivo informático (laptop, celular, etc.) que maneja los recursos de hardware y provee servicios a los programadores de aplicaciones mediante interrupciones de software. Los sistemas operativos se ejecutan en modo privilegiado, aunque puede que una parte se ejecute en modo usuario.\\
El núcleo o kernel es la parte más fundamental del sistema operativo, que se encarga de hacer los primeros pasos para el arranque del sistema y de proveer las rutinas de atención a interrupciones de software, que utilizan los programadores de aplicaciones.\\
El kernel se define como la parte del sistema operativo que se ejecuta en modo privilegiado.\\
En este trabajo educativo nos proponemos hacer un sistema operativo muy básico para entender los fundamentos de la programación de sistemas operativos.\\
Para ello completamos diversos fragmentos de un esqueleto de kernel provisto por los docentes. 
