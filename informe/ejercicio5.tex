\subsection{Ejercicio 5}

\subsubsection{IDT}
Inicializamos tres nuevas entradas en la $idt$: la 32, 33 y 70. Las dos primeras con campo $dpl$ en $0x0$, indicando que pueden ser accedidas por el kernel solamente, y la última con $dpl$ en $0x3$, para que pueda
ser accedida por los usuarios.\\
Debemos remapear las direcciones de las interrupciones 32 y 33 porque caen en índices de excepciones del
procesador. Para esto llamamos a función $resetear$ $pic$. Luego habilitamos las interrupciones externas 
llamando a la función $habilitar$ $pic$ que setea flag en registro de control.\\
Para las rutinas de atención 32, 33 y 70 debemos resguardar los registros de uso general de manera que 
las interrupciones sean transparentes a la tarea interrumpida. Luego, excepto en la rutina 70, se debe comunicar al $PIC$ que la interrupción
 será atendida. Esto se hace llamando a la función $fin$ $intr$ $pic1$. Al salir de la rutina se debe restaurar
los registros y ejecutar la instrucción $iret$.\\
La interrupción del reloj solo debe invocar una función, en este caso $game$ $tick$. En cambio la rutina 
del teclado debe leer del puerto $0x60$ la información de la tecla. Esto se hace con la instrucción $in$ en un
registro, por ejemplo $in$ $ax$ , $0x60$.

